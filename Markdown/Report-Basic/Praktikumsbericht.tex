% Options for packages loaded elsewhere
\PassOptionsToPackage{unicode}{hyperref}
\PassOptionsToPackage{hyphens}{url}
\PassOptionsToPackage{dvipsnames,svgnames*,x11names*}{xcolor}
%
\documentclass[
  10pt,
  ngerman,
]{article}
\usepackage{lmodern}
\usepackage{amssymb,amsmath}
\usepackage{ifxetex,ifluatex}
\ifnum 0\ifxetex 1\fi\ifluatex 1\fi=0 % if pdftex
  \usepackage[T1]{fontenc}
  \usepackage[utf8]{inputenc}
  \usepackage{textcomp} % provide euro and other symbols
\else % if luatex or xetex
  \usepackage{unicode-math}
  \defaultfontfeatures{Scale=MatchLowercase}
  \defaultfontfeatures[\rmfamily]{Ligatures=TeX,Scale=1}
\fi
% Use upquote if available, for straight quotes in verbatim environments
\IfFileExists{upquote.sty}{\usepackage{upquote}}{}
\IfFileExists{microtype.sty}{% use microtype if available
  \usepackage[]{microtype}
  \UseMicrotypeSet[protrusion]{basicmath} % disable protrusion for tt fonts
}{}
\makeatletter
\@ifundefined{KOMAClassName}{% if non-KOMA class
  \IfFileExists{parskip.sty}{%
    \usepackage{parskip}
  }{% else
    \setlength{\parindent}{0pt}
    \setlength{\parskip}{6pt plus 2pt minus 1pt}}
}{% if KOMA class
  \KOMAoptions{parskip=half}}
\makeatother
\usepackage{xcolor}
\IfFileExists{xurl.sty}{\usepackage{xurl}}{} % add URL line breaks if available
\IfFileExists{bookmark.sty}{\usepackage{bookmark}}{\usepackage{hyperref}}
\hypersetup{
  pdflang={de-De},
  colorlinks=true,
  linkcolor=Maroon,
  filecolor=Maroon,
  citecolor=Blue,
  urlcolor=Blue,
  pdfcreator={LaTeX via pandoc}}
\urlstyle{same} % disable monospaced font for URLs
\usepackage[margin=1in]{geometry}
\usepackage{longtable,booktabs}
% Correct order of tables after \paragraph or \subparagraph
\usepackage{etoolbox}
\makeatletter
\patchcmd\longtable{\par}{\if@noskipsec\mbox{}\fi\par}{}{}
\makeatother
% Allow footnotes in longtable head/foot
\IfFileExists{footnotehyper.sty}{\usepackage{footnotehyper}}{\usepackage{footnote}}
\makesavenoteenv{longtable}
\usepackage{graphicx}
\makeatletter
\def\maxwidth{\ifdim\Gin@nat@width>\linewidth\linewidth\else\Gin@nat@width\fi}
\def\maxheight{\ifdim\Gin@nat@height>\textheight\textheight\else\Gin@nat@height\fi}
\makeatother
% Scale images if necessary, so that they will not overflow the page
% margins by default, and it is still possible to overwrite the defaults
% using explicit options in \includegraphics[width, height, ...]{}
\setkeys{Gin}{width=\maxwidth,height=\maxheight,keepaspectratio}
% Set default figure placement to htbp
\makeatletter
\def\fps@figure{htbp}
\makeatother
\setlength{\emergencystretch}{3em} % prevent overfull lines
\providecommand{\tightlist}{%
  \setlength{\itemsep}{0pt}\setlength{\parskip}{0pt}}
\setcounter{secnumdepth}{5}
\ifxetex
  % Load polyglossia as late as possible: uses bidi with RTL langages (e.g. Hebrew, Arabic)
  \usepackage{polyglossia}
  \setmainlanguage[]{german}
\else
  \usepackage[shorthands=off,main=ngerman]{babel}
\fi

\title{Praktikum im Leibniz-Institut für Präventionsforschung und
Epidemiologie}
\usepackage{etoolbox}
\makeatletter
\providecommand{\subtitle}[1]{% add subtitle to \maketitle
  \apptocmd{\@title}{\par {\large #1 \par}}{}{}
}
\makeatother
\subtitle{Praktikumsbericht}
\author{Zehui Bai\\
Medical Biometry/Biostatistics\\
Betreut von Dr.~Ronja Foraita\\
Matrikelnummer:3183673\\
Zeitraum: 01.08.2019 -- 12.09.2019}
\date{}

\begin{document}
\maketitle

{
\hypersetup{linkcolor=}
\setcounter{tocdepth}{2}
\tableofcontents
}
\hypertarget{praktikumsstelle}{%
\section{Praktikumsstelle}\label{praktikumsstelle}}

\hypertarget{vorstellung-der-praktikumsstelle}{%
\subsection{Vorstellung der
Praktikumsstelle}\label{vorstellung-der-praktikumsstelle}}

Das Leibniz-Institut für Präventionsforschung und Epidemiologie -- BIPS
wurde 1981 als „Bremer Institut für Präventionsforschung und
Sozialmedizin (BIPS)`` gegründet. Es ist damit eines der ältesten
Epidemiologie-Institute Deutschlands.

BIPS ist ein unabhängiges, interdisziplinär arbeitendes
Epidemiologie-Forschungsinstitut, das seine Aufgabe in der Erforschung
von Krankheitsursachen und der Vorbeugung gegen Erkrankungen sieht. Das
Institut entwickelt mit seiner Forschung wirksame Strategien zur
Prävention chronischer nichtübertragbarer Erkrankungen. Das BIPS
erforscht Ursachen für Gesundheitsstörungen und entwickelt neue
Konzepte, um Krankheiten vorzubeugen. Es untersucht, wie wirksam diese
Maßnahmen sind und stellt die Forschungsergebnisse der Öffentlichkeit
zur Verfügung. Neben der epidemiologischen Forschung sichert das
Institut den Transfer seiner Forschungsergebnisse in die Praxis. Es
entwickelt Präventionsprogramme und berät Entscheidungsträger auf
nationaler und internationaler Ebene (BIPS-Allgemein, 01.10.2019).

\begin{verbatim}
\begin{center}
\includegraphics[width=4.8in]{Fotos/Organigramm}
\begin{figure}[!ht]
\caption{Organisationsstruktur in BIPS}
\end{figure}
\end{center}
\end{verbatim}

\begin{longtable}[]{@{}l@{}}
\caption{Datenstruktur von GWAS}\tabularnewline
\toprule
Variablensname\tabularnewline
\midrule
\endfirsthead
\toprule
Variablensname\tabularnewline
\midrule
\endhead
ID\_NO\tabularnewline
Material\tabularnewline
Spitted\tabularnewline
Datum\tabularnewline
GAWS\tabularnewline
Projekt\_Name\tabularnewline
extraConcentration\tabularnewline
country\tabularnewline
DNA\_extraction\tabularnewline
\bottomrule
\end{longtable}

\hypertarget{literatur}{%
\section{Literatur}\label{literatur}}

{[}1{]} BIPS-Allgemein.(Tag des Zugriffs: 01.10.2019)

\url{https://www.bips-institut.de/das-institut/ueber-das-institut.html}

\end{document}
